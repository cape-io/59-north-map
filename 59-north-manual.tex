
\documentclass[11pt]{article}
\usepackage{softwarespec}
\projectname{Lifting Bridges}

\title{59-north.com Mapbox Integration Usage Manual}

\author{Jordan Bettis}

\date{\today{}}
\documentnumber{59-NORTH-02}

\begin{document}

\maketitle
\tableofcontents

\newpage

\section{Mapbox Account Changes}

\subsection{Install the Style in your Mapbox account}

Log into your mapbox account.

\begin{enumerate}
\item Choose the \texttt{Styles} page in the left menu.
\item Click the \texttt{New Style} button in the header.
\item Click \texttt{Upload a Style} in the top right of the popup.
\item Click on \texttt{Select a file}.
\item Choose \texttt{style/style.json} from the repository.

\end{enumerate}

You will now find yourself in the style editor viewing the style you
just uploaded. Look to the top left of the window. There is a button
titled \texttt{Publish} and a button to the right of it with an arrow
pointing up and to the right, with the tool tip \emph{View Style
  Details}. Press that button.

Now look for the ``Style URL'' in the right column. Make a note of
this. \emph{You will need this URL later.}

\label{style-url}

\subsection{Get an Access Token}

Click on \texttt{Home} in the left navigation menu, then click on
\texttt{My access tokens}. There should be a Default Public Token. If
there is, make a note of it. If there is not click \texttt{Create a
  new token}. Make sure all the checkmarks are checked in the
\texttt{Public Scopes} section.

Make a note of the token. You will also need this later.

\label{access}

\section{Setup on Squarespace}

\subsection{Add the Header Code Injection}

The project proposal specified a customization of your template to
inject the custom code needed to initialize the map on your
site. While working on the project, I became concerned that this
method would impact the future upgradability of your site. You would
be unable to upgrade or change your theme in the future because you
would be ``pegged'' to the customized them I would have created.

I decided to look for other ways to inject the custom code I wrote
into your site. I found that Squarespace has an advanced code
header/footer injection feature for commercial sites. I therefore
built the custom script so that it can be integrated using this
feature.

The custom code is available in this repository as
\texttt{createMap.js} but you won't actually need that file. I have
also provided a compressed version of the file---embedded in some
HTML---in the \texttt{header-code-injection.html} file.

You will need to open this file in a text editor. If you don't know
how to do that, see the appendix (Section \ref{texteditor}).

In your squarespace admin:

\begin{enumerate}
\item Click on \texttt{Settings} in the left sidebar.
\item Click on the last item under \texttt{Website}, labeled \texttt{Advanced}.
\item Click on the \texttt{Code Injection} option.
\item Copy the contents of the \texttt{header-code-injection.html}
  file and paste it into the \texttt{Header} input box.
\item Click \texttt{Save} in the top left. 
\end{enumerate}

Your squarespace site is now setup to load maps from Mapbox.

\section{Create a Map}

\subsection{GeoJSON}

Mapbox uses the GeoJSON\footnote{http://geojson.org} format to
represent map data. JSON\footnote{http://json.org} stands for
JavaScript Object Notation. It is a structured text format. Structured
text is a way to represent data that is easy for both humans and
computers to understand. GeoJSON, therefore, is geospatial data
represented as structured text using the JSON standard.

I have included the GeoJSON I used to create the map from Sweden to
Scotland in the \texttt{scotland.geojson} file. It's worth taking a
careful look at it. Below, I'm going to show how to use a tool
provided by Mapbox to create these files, but it will be useful to
understand the structure of the files.

The Scotland file contains two \emph{Points} and a multi-segment
\emph{LineString}. The points represent destinations and the line
string represents the path between the destinations.

One thing worth noting is that the coordinates are specified as
latitude/longitude pairs, but that the order is reversed. \emph{In
  GeoJSON, the longitude is always specified first, then the
  latitude.} Each point has just one coordinate, but the multi-line
has an array of points. I started the line in the east progressing
west, so you'll notice a monotonically declining longitude.

Of particular importance is properties attached to each
feature. Consider the list of properties attached to the Orkney
Islands point:

\begin{verbatim}

    "properties": {
        "title": "Orkney Islands",
        "description": "An archipelago in the Northern Islands of Scotland",
        "noonsite": "http://www.noonsite.com/Countries/UnitedKingdom",
        "wikipedia": "https://en.wikipedia.org/wiki/Orkney"
    },
\end{verbatim}

\emph{The \texttt{title} field is required.} Each point \emph{must}
have a title. The description is an optional text field, it should be
a sentence giving some more detail about the destination.

You can add an arbitrary number of additional properties. In this
example I added ``noonsite'' and ``wikipedia.'' Each additional
property, in addition to \texttt{title} and \texttt{description}, is
assumed to be a link, with the key as a link title and the value as
the URL. These will be displayed in the popup, below the description.

The properties for a line string are much more limited:

\begin{verbatim}
    "properties": {
        "link": "https://www.59-north.com/arctic-passages/2018/
5/1/1-smogen-sweden-orkney-scotland"
    },
\end{verbatim}

There is only one property, named \texttt{link}. The value should be a
URL to the sign up page.

One last observation about this GeoJSON file: note that each feature
has an \texttt{id} field. The only requirement for this field is one
must exist for each feature and its value must be unique in the
document.

\subsection{Using Mapbox to create GeoJSON files}

\label{geojson}

Mapbox has a data editor that you can use to build GeoJSON documents
by using drawing tools. Go to your Mapbox account and select
\texttt{Datasets} in the left nav menu. Then click \texttt{New
  dataset}. You'll be asked to give the dataset a name. Choose
something sensible.

You'll now see the data editor console. A very useful feature is the
ability to search the map for locations. For instance, I'd like to
make a map of a route from Chicago to Muskegon Michigan. I start by
going to the top right sidebar and typing ``Chicago'' into the
\texttt{Search places} input box. This zooms and pans the map to
Chicago. I zoom in a little bit to see Monroe Harbor.

Near the top of the left sidebar is a button with an upside-down
teardrop on it. This is the ``Point'' tool. I click on it and then add
a point by left-clicking in Monroe Harbor. After I add the point, the
left sidebar switches to property editor. The point must have a title,
so I take care of that first, clicking \texttt{Add new}. I type
``title'' in the \texttt{field} input box and write ``Chicago'' in the
\texttt{value} input, and then click \texttt{Confirm}.

The title is the only required field, but you can do the same for
\texttt{description} and any other field. Of important note is that
\texttt{title} and \texttt{description} must be lower-case. The same
is true for \texttt{link} on the line string.

I now need to draw a line-string from Chicago to Muskegon. I click the
icon that looks like a barbell to the right of the point
teardrop. Then I left-click on the map near the point I added, and
then continue left-clicking to show the path out of the harbor. Once
my line string gets to open water, I want to get a zoomed-out view so
I press the minus key ``-'' several times. I can also use the arrow
keys on the keyboard to pan without risking adding accidental
segments.

When the line gets to Muskegon I zoom back in by pressing the equals
``='' key on the keyboard. I double click to add the last point in
Muskegon Harbor to end the path.

Each path needs a \texttt{link} property. That is added as
above. After adding another point for Muskegon I am now ready to
download the GeoJSON.

To do this, click the \texttt{Save} button on the top of the left
sidebar. Then \texttt{View dataset details} below the export
button. You will not need to use the export button.

Now click the \texttt{Download as GeoJSON} link in the right
sidebar. Save the file to your hard drive.

\subsection{Adding a Map to a Squarespace Page}

Now it all comes together. You have:

\begin{itemize}
\item Your style URL (Section \ref{style-url}).
\item Your access token (Section \ref{access}).
\item The GeoJSON file (Section \ref{geojson}).
\item The HTML snippet code in the \texttt{snippet.html} file.

\end{itemize}

Add a \texttt{code} block to the Squarespace page where you want the
map to appear. Copy and paste the contents of \texttt{snippet.html}
into the input. Make sure the type is ``HTML'' and the \texttt{Display
  Source} checkbox is \emph{unchecked}.

Copy and paste your style URL into the snippet replacing the text
``REPLACE THIS WITH YOUR STYLE URL (will be
mapbox://styles/something/something)''. Be sure to preserve the quote
marks around it. For me, the resulting line looks like this:

\begin{verbatim}

    style: 'mapbox://styles/jordanb/cj7zhvf336t2u2rlmc0zyt82f',
  
\end{verbatim}

Do the same with the token, again, preserving the quote marks around it.

Next you need to open the downloaded JSON document in your text
editor. Copy the \emph{entire} file and paste it into the snippet,
replacing the value:

\begin{verbatim}

    {'REPLACE THIS': 'WITH YOUR GEOJSON'}

\end{verbatim}

Note that the GeoJSON must \emph{not} be in quotes, unlike the URL and
access token. The line will look something like this:

\begin{verbatim}

    var mapData = {
      "features": [
        {
          "type": "Feature",
====== LOTS OF OMITTED LINES OF CODE ======
        }
      ]
    };

\end{verbatim}

Once you've done this, you can save the page. You're done! You
probably won't be able to view the map while logged-in due to security
restrictions, but you can see the map by opening a private/incognito
browsing window and viewing the page with it.

If you have any trouble or if anything is unclear please reach out to
me.

\appendix

\section{Text Editor}
\label{texteditor}

You must use a text editor to open any of the downloaded files,
including \texttt{snippet.html}, \texttt{header-code-injection.html}
or any of your downloaded GeoJSON documents. On Windows, you can use
\texttt{Notepad}. On a Mac, things are more tricky. You can use the
\texttt{TextEdit} program, but it \emph{must be in plain-text
  mode}. You'll have to lookup how to do that.

If you can't get TextEdit working, you can also download or buy a text
editor. Some examples are \texttt{Atom}, \texttt{Sublime Text}, or
\texttt{TextMate}.

If you're having difficulty opening a file in a text editor, you may
want to try changing the file's extension to ``.txt''. So for
instance, \texttt{snippet.html} would become \texttt{snippet.txt}.


\end{document}
